\documentclass{article}
\usepackage{amsmath}
\begin{document}
\(h\) ist auch in der Heisenbergschen Unschärferelation zentral. Sie legt fest,
dass es eine Grenze für die Genauigkeit gibt, mit der Paare von komplementären
Größen, wie Position und Impuls, gleichzeitig bestimmt werden können:
\[
\Delta x \cdot \Delta p \geq \frac{h}{4 \pi}
\]

**Wellen-Teilchen-Dualität und De-Broglie-Wellen**

**Wellen-Teilchen-Dualität:**
Nach der De-Broglie-Hypothese hat jede Materie sowohl Wellen- als auch
Teilcheneigenschaften. Die Wellenlänge \(\lambda\) eines Teilchens hängt vom
Impuls \(p\) ab und wird durch die Formel:
\[
\lambda = \frac{h}{p}
\]
beschrieben, wobei \(p = mv\) (Impuls = Masse \(\times\) Geschwindigkeit) gilt.

**Experimentelle Bestätigung:**
Die Elektronenbeugungsexperimente zeigen, dass Elektronen, obwohl sie als
Teilchen angesehen werden, Interferenz- und Beugungsmuster erzeugen können,
was ihre Welleneigenschaften bestätigt.

**Photonen und ihre Eigenschaften**

**Energie und Masse eines Photons:**
Ein Photon hat keine Ruhemasse (\(m_0 = 0\)), aber seine Energie ist
durch
\[
E = h \nu
\]
bestimmt. Die dynamische Masse eines Photons kann aus der Gleichung
\[
E = mc^2
\]
abgeleitet werden.
\end{document}
